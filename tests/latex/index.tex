\hypertarget{index_intro_sec}{}\section{Introduction}\label{index_intro_sec}
This directory contains the unit tests for the mopatop model \hypertarget{index_Main}{}\subsection{Main ideas}\label{index_Main}
The system uses the cppunit test system, version 1.\+14 to test each part of the model against some standard cases~\newline
For each class in the model, a header file is defined here containing tests of that class. Each header containes~\newline
a test suite, each of which runs through some of the methods defined in the classes to check that behave as expected~\newline
in a well defined and limited number of cases where the output can be known ahead of time. This helps to prevent some~\newline
of the more easily caught programming or conceptual errors. Each test suite is added to a test\+Runner defined in \mbox{\hyperlink{testing_8cpp}{testing.\+cpp}} ~\newline
Note that because some of the classes can depend on each other, running some of the tests might potentially affect others~\newline
(e.\+g. the time\+Step class is static, so if its parameters are altered, subsequent tests that depend on the timestep might~\newline
be altered). This should be borne in mind when writing new tests.~\newline
When modifications are made to any of the classes, tests should be re-\/run to chekc that nothing has been broken. Any new~\newline
code not subject to testing should have tests added here as necessary. The tests themsevles have their own documentation~\newline
generated by Doxygen and kept in the html sub-\/directory. \hypertarget{index_Compiling}{}\subsection{Compiling}\label{index_Compiling}
On a linux system with g++ installed just do ~\newline
g++ -\/otests -\/fopenmp -\/std=c++17 -\/lcppunit ../timestep.cpp ../places.cpp ../disease.cpp ../schedulelist.cpp ../agent.cpp \mbox{\hyperlink{testing_8cpp}{testing.\+cpp}}~\newline
Note the current version requires g++$>$= and c++17 in order for the filesystem function to work (for creating new directories etc.)~\newline
\hypertarget{index_Run}{}\subsection{Running}\label{index_Run}
At present this is a simple command-\/line application -\/ just type the executable name (tests above) and then return.~\newline
This will use the parameter file called test\+Parameter\+File in this directory. ~\newline

\begin{DoxyCode}
./tests
\end{DoxyCode}
 The parameter file allows for a number of tests that depend on parameter settings. \hypertarget{index_Details}{}\subsection{Details}\label{index_Details}
A test\+Runner is defined from the Cpp\+Unit\+::\+Text\+Ui\+::\+Test\+Runner class. Each of the testsuites is added to this using 
\begin{DoxyCode}
runner.addTest( testClass::suite() );
\end{DoxyCode}
 where test\+Class is defined in one of the header files. Each header file uses the C\+P\+P\+U\+N\+I\+T\+\_\+\+T\+E\+S\+T\+\_\+\+S\+U\+I\+TE macro to define a testsuite~\newline
which is a set of functions, each function containing one or more tests 
\begin{DoxyCode}
 CPPUNIT\_TEST\_SUITE( testClass );
\textcolor{comment}{//add tests defined below}
CPPUNIT\_TEST( function1 );
CPPUNIT\_TEST( function2 );
CPPUNIT\_TEST\_SUITE\_END();
\end{DoxyCode}
 If needed a set\+Up() and tear\+Down() function can be added, which will automatically create and destroy variables afresh at the start of~\newline
each test function, provided the class inherits from Cpp\+Unit\+::\+Test\+Fixture e.\+g. 
\begin{DoxyCode}
\textcolor{keyword}{class }\mbox{\hyperlink{classplaceTest}{placeTest}} : \textcolor{keyword}{public} CppUnit::TestFixture  \{
  place* \mbox{\hyperlink{classplaceTest_a0f4e660bdadc034488490bbadc33e09f}{p}};
\textcolor{keyword}{public}:
 \textcolor{comment}{//persistent objects to use during testing}
 \textcolor{keywordtype}{void} \mbox{\hyperlink{classplaceTest_a1818bbe31325c6d9344b35f3b510cacf}{setUp}}()
 \{
     \mbox{\hyperlink{classplaceTest_a0f4e660bdadc034488490bbadc33e09f}{p}} = \textcolor{keyword}{new} place();
 \}
 
 \textcolor{keywordtype}{void} \mbox{\hyperlink{classplaceTest_a2a0d52566e44cc385dc8fecdae504d6f}{tearDown}}() 
 \{
     \textcolor{keyword}{delete} \mbox{\hyperlink{classplaceTest_a0f4e660bdadc034488490bbadc33e09f}{p}};
 \}
 \};
\end{DoxyCode}
 NB these function defintions must come B\+E\+F\+O\+RE the above testsuite defintions.~\newline
In the test function typically asserts are used to check the result of of a test e.\+g. 
\begin{DoxyCode}
\textcolor{keywordtype}{void} function1()
\{
    \textcolor{comment}{//default ID is zero}
    CPPUNIT\_ASSERT(0==p->getID());
    \}
\end{DoxyCode}
 These are pass/fail tests, and will be reported giving the text of the failed test and line number in the code if needed~\newline
In addition tests may report text to stdout where required. Note the uses here of the pre-\/defined pointer p from the set\+Up().~\newline
Variables can also be defined as needed within each test function however.~\newline
An output directory is generated afresh (and any previous values are overwritten) each time the tests are run contains ~\newline
 a small amount of data for checiking output works as expected.. 